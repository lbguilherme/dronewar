

\section{Mesh}
\subsection{Ray Cast}
A ray has direction $\mathbf{\hat n}$ and source $\mathbf l$. An edge is defined $\mathbf e_{ij} = \mathbf r_j - \mathbf r_i$. A triangle $T$ has edges $\mathbf e_{ij}$ and $\mathbf e_{ik}$. Let $\mathbf p_0$ be the vertex $\mathbf r_i$ on $B$. 
$$ \mathbf l + \mathbf{\hat n}t = \mathbf p_0 + \mathbf e_{ij}u + \mathbf e_{ik}v $$
$$ \mathbf l - \mathbf p_0 =  -\mathbf{\hat n}t + \mathbf e_{ij}u + \mathbf e_{ik}v $$

In matrix notation:
$$ \mathbf l - \mathbf p_0 = 
 \begin{bmatrix}
  -\mathbf{\hat n} & \mathbf e_{ij} & \mathbf e_{ik}
 \end{bmatrix}
 \begin{bmatrix}
  t \\
  u \\
  v
 \end{bmatrix}
$$

Therefore: $P = Ax$ and the solution: $x = A^{-1} P$. \\

Once a cyclic permutation is peformed, the component vectors transform:
$$
\mathbf r_i + u \mathbf e_{ij} + v \mathbf e_{ik} = \mathbf r_j + u' \mathbf e_{ji} + v' \mathbf e_{jk}
$$
$$
-u' \mathbf e_{ij} + v' \mathbf e_{jk} = \left(u+v-1\right)\mathbf e_{ij} + v \mathbf e_{jk}
$$
\newpage


% ====================================================================


\section{Solid}
\subsection{Volume of a solid}
A triangle $T$ in a solid $V$ with boundary $\partial V$ has vertices $\mathbf r_i$. Edges are defined $\mathbf e_{ij} = \mathbf r_j - \mathbf r_i$. The volume of a solid can be calculated using volumetric contributions from each triangle $T_n$.
$$
V_n = \iint_{\partial V}\mathbf F\cdot d\mathbf{s} = 
\iint_{\Delta_n}\mathbf F(\mathbf r(s, t))\cdot\mathbf{\hat n}\left|\frac{\partial\mathbf r}{\partial s}\times\frac{\partial\mathbf r}{\partial t} \right| ds\mbox{ }dt
$$
$$
V_n = \iint_{\Delta_n}\mathbf F(\mathbf r(s, t))\cdot\mathbf{\hat n}A_{tr}\mbox{ }ds\mbox{ }dt =
\frac{1}{2}\iint_{\Delta_n}\mathbf F(\mathbf r(s, t))\cdot\mathbf A \mbox{ }ds\mbox{ }dt
$$

Chosing $F$:
$$
\sum_i \frac{\partial F_i}{\partial x_i} = 1 \quad\Longleftarrow\quad
\mathbf F(\mathbf r) = \frac{\mathbf r}{3}, \quad\mbox{ for }\quad
\mathbf F = \sum_i F_i \mathbf e_i
$$

We can define $\mathbf r(s, t) = \mathbf v_1(t) + \mathbf v_2 (s)$. Reminding only cyclic permutations on $ijk$ are only applied, we define:
\[
  \begin{cases}
   \mathbf v_1(t) = \mathbf r_i + \mathbf e_{ij}t  \\
   \mathbf v_2(s) = \mathbf r_j + \mathbf e_{jk}s
  \end{cases}
  \quad\Longrightarrow\quad
  \frac{\partial\mathbf r}{\partial t} = \frac{\partial\mathbf v_1}{\partial t} = \mathbf e_{ij}, \quad\quad
  \frac{\partial\mathbf r}{\partial s} = \frac{\partial\mathbf v_2}{\partial s} = \mathbf e_{jk}
\]

The integral then resumes:
$$
V_n = \frac{1}{6}\iint_{\Delta_n}\left( \mathbf v_1(t) + \mathbf v_2(s)\right)\cdot\mathbf A \mbox{ }ds\mbox{ }dt = 
\left(\mathbf r_i + \mathbf r_j\right)\cdot\frac{\mathbf A}{6} + \frac{\mathbf A}{6}\cdot\int_0^1\int_0^1\left( t \mathbf e_{ij} + s \mathbf e_{jk} \right) \mbox{ }ds\mbox{ }dt
$$

The volume of a solid is then:
$$
V = \frac{1}{6}\sum_n\mathbf A_n\cdot\left[ \mathbf r_i + \mathbf r_j + \frac{\mathbf e_{ij} + \mathbf e_{jk}}{2}  \right]_n
$$
\newpage

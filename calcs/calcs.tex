\documentclass[10.5pt,oneside]{book}
\addtolength{\evensidemargin}{-1.5cm}
\addtolength{\oddsidemargin}{-1,5cm}
\addtolength{\textwidth}{3cm}
\addtolength{\textheight}{2cm}
%\addtolength{\topmargin}{-1cm}
%\addtolength{\headheight}{-2cm}

\usepackage[english]{babel}
\usepackage{amssymb}
\usepackage[utf8]{inputenc}
\usepackage{amsfonts}
\usepackage[dvips]{graphicx}
\usepackage{latexsym}
%\usepackage[all]{xy}
\usepackage{amsmath}
\usepackage[normalem]{ulem}
\usepackage{palatino}
\usepackage{enumerate}
%\usepackage{makeidx}
\usepackage{amssymb,amsthm,amsfonts,amsmath,pifont}
\usepackage{pstricks}
\usepackage{graphicx}

\usepackage{hyperref}

\setlength{\marginparwidth}{0pt}


%% Comandos relacionados à trigonometria e números complexos
\renewcommand{\cot}{\,\mbox{$\rm{cotg}\,$}}
\newcommand{\cis}{\,\mbox{$\rm{cis}\,$}}
\newcommand{\erf}{\,\mbox{$\rm{erf}\,$}}
\newcommand{\erfi}{\,\mbox{$\rm{erfi}\,$}}

\begin{document}
\title{The Calculations}
\author{Phy =D}
\maketitle

\hypersetup{
    colorlinks=false, %set true if you want colored links
    linktoc=all,     %set to all if you want both sections and subsections linked
}
\tableofcontents

\chapter{Math}
\section{Solid}
\subsection{Volume of a solid}
Assume surface triangles defined by 3 vertexes $\mathbf r_k$ where edges are defined $\mathbf {e_{ij}} = \mathbf r_i - \mathbf r_j$. The volumetric contribution:
$$
V_n = \iint_{\Delta_n} F(\mathbf r)\mathbf{\hat n}\cdot\mathbf{dS} = 
\iint_{\Delta_n} F(\mathbf r(s, t))\left|\frac{\partial\mathbf r}{\partial s}\times\frac{\partial\mathbf r}{\partial t} \right| ds\mbox{ }dt
$$

Chosing $F$:
$$
\sum_i \frac{\partial F}{\partial x_i}n_i = 1 \quad\Longleftarrow\quad
\frac{\partial F}{\partial x}n_1 = 1 \quad\Longleftarrow\quad
F(x_k) = \frac{x}{\mathbf{\hat i}\cdot\mathbf{\hat n}}
$$


Defining $\mathbf r(s, t) = \mathbf v_1(t) + \mathbf v_2 (s)$, and defining:
\[
  \begin{cases}
   \mathbf v_1(t) = \mathbf r_1 + \mathbf e_{21}t  \\
   \mathbf v_2(s) = \mathbf r_2 + \mathbf e_{32}s
  \end{cases}
  \quad\Longrightarrow\quad
  \frac{\partial\mathbf r}{\partial t} = \frac{\partial\mathbf v_1}{\partial t} = \mathbf e_{21}, \quad\quad
  \frac{\partial\mathbf r}{\partial s} = \frac{\partial\mathbf v_2}{\partial s} = \mathbf e_{32}
\]

Then:
$$
V_n = \mathbf{\hat i}\cdot\frac{\mathbf r_1 + \mathbf r_2}{\mathbf{\hat i}\cdot\mathbf{\hat n}}
\int_0^1\int_0^1 \left(\frac{\mathbf e_{21}t + \mathbf e_{32}s}{\mathbf{\hat i}\cdot\mathbf{\hat n}}\cdot\mathbf{\hat i}\right) \frac{\left|\mathbf e_{21}\times\mathbf e_{32} \right|}{2} ds\mbox{ }dt
$$
$$
V_n = \mathbf{\hat i}\cdot\frac{\mathbf r_1 + \mathbf r_2}{\mathbf{\hat i}\cdot\mathbf{\hat n}} 
\frac{\left|\mathbf e_{21}\times\mathbf e_{32} \right|}{2 \mathbf{\hat i}\cdot\mathbf{\hat n}}\mathbf{\hat i}\cdot
\int_0^1\int_0^1 \left(\mathbf e_{21}t + \mathbf e_{32}s\right) ds\mbox{ }dt
$$
$$
V_n = \mathbf{\hat i}\cdot\frac{\mathbf r_1 + \mathbf r_2}{\mathbf{\hat i}\cdot\mathbf{\hat n}} 
\frac{\left|\mathbf e_{21}\times\mathbf e_{32} \right|}{2 \mathbf{\hat i}\cdot\mathbf{\hat n}}\mathbf{\hat i}\cdot
\left(\frac{\mathbf e_{21} + \mathbf e_{32}}{2}\right)
$$

Therefore, the volume of a solid:
$$
V =  \frac{1}{4} \sum_n
\left[ \mathbf{\hat i}\cdot \left( \frac{\mathbf r_1 + \mathbf r_2}{\mathbf{\hat i}\cdot\mathbf{\hat n}}\right) \right]
\left[ \mathbf{\hat i}\cdot \left( \frac{\mathbf e_{21} + \mathbf e_{32}}{\mathbf{\hat i}\cdot\mathbf{\hat n}}\right) \right]
\left|\mathbf e_{21}\times\mathbf e_{32} \right|
$$



\end{document}
